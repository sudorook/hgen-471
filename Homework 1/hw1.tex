%!TEX encoding = UTF-8 Unicode

\documentclass{essay}
\usepackage[margin=1in]{geometry}
\usepackage{mathtools}
\usepackage{minted}

\begin{document}
\sffamily

\setsmallheader{Homework 1}{HGEN 47100}{Ansel George}
\setbodyheader{Ansel George}{Homework 1}

\begin{essaystyle}

\essaysection{Problem Set 1}
\vspace{.5em}

\textbf{1.} Early onset Alzheimer’s disease is very rare; for illustrative
purposes, assume it is 0.1\% among adults aged 30--60. Rare variants in 3 genes,
APP, PSEN1 and PSEN2 have been identified as causing early onset AD in a
dominant fashion, With $P(AD | \textrm{any of the three variants}) = 1$. Early onset AD
can also be caused by head injury; many other non-genetic factors have been
suggested. In a series of 101 cases of early onset AD, only 7 (or approximately
7\%) were found to have these variants in APP, PSEN1 or PSEN2; that is, the
attributable risk due to the three rare variants is low. For simplicity, assume
that the probability of variants in these 3 genes is so rare that we can assume
$P(\textrm{no variant in any gene}) \approx 1$. Let the disease allele D symbolize a
variant in any one of the three genes, d is no variant, and Y=1 means AD
present. Estimate the probability of a phenocopy, $P(Y=1 | dd)$ (also known as
phenocopy rate) for these genes combined, using the data given and Bayes Rule.


$P(AD) = .001$ \\
$P(\textrm{any of the three variants} | AD) = P(D|AD) = \frac{7}{101} = 0.06930693$ \\
$P(\textrm{no variant in any gene}) = P(\overline{D}) = P(dd) \approx 1$\\

\begin{align}
  P(Y=1 | dd) &= P(AD | dd) = \frac{P(dd|AD)P(AD)}{P(dd)} \\
  &= \frac{(1-\frac{7}{101}).001}{1} \\
  &\approx .000931
\end{align}


\textbf{2.} Suppose we are dealing with a quantitative recessive trait, which
is distributed as $N(\mu,1)$ when there are two variants, and $N(0,1)$ otherwise.
Calculate the probability that a randomly selected person with two variants has
a trait higher than a person with one or no variants, when $\mu=0.5$, and when
$\mu=2$.

Let $X_{1} \sim \mathrm{N}(\mu,1)$ and $X_{2} \sim \mathrm{N}(0,1)$.

\begin{align}
  p(X_{1} > X_{2}) &= p(X_{1} - X_{2} > 0) \\
  &= p(Z > 0)
\end{align}

The new variable $Z = X_{2} - X_{1}$ follows normal distribution
$\mathrm{N}(\mu - 0, 1 + 1)=\mathrm{N}(\mu, 2)$. The variances can simply be
added because they are equivalent for $X_{1}$ and $X_{2}$.

Therefore, to find $p(Z > 0)$, use the CDF for a Normal distribution (F).

\begin{align}
  p(Z > 0) &= 1 - p(Z < 0) \\
  &= 1 - p(Z < 0) \\
  &= 1 - F(\mu, 2)
\end{align}

For $\mu=.5$:

\begin{minted}{r}
1 - pnorm(0, mean=.5, sd=sqrt(2))
[1] 0.6381632
\end{minted}

For $\mu=2$:

\begin{minted}{r}
1 - pnorm(0, mean=2, sd=sqrt(2))
[1] 0.9213504
\end{minted}

\textbf{3.} Assume the genotypes AA, Aa and aa have frequencies u, v and w in a
randomly mating population. By considering all possible outcomes of all
possible mating types, show that the offspring genotypes follow HWE with the
same allele frequencies in both generations.

\begin{tabular}{ l | c c c }
  & AA & Aa & aa \\
  \hline
  AA & AA & \begin{tabular}{@{}c@{}} $\frac{1}{2}$AA \\ $\frac{1}{2}$Aa \end{tabular} & Aa \\
  Aa & \begin{tabular}{@{}c@{}} $\frac{1}{2}$AA \\ $\frac{1}{2}$Aa \end{tabular}  & \begin{tabular}{@{}c@{}} $\frac{1}{4}$AA \\ $\frac{1}{2}$Aa \\ $\frac{1}{4}$Aa \end{tabular} & \begin{tabular}{@{}c@{}} $\frac{1}{2}$Aa \\ $\frac{1}{2}$aa \end{tabular}  \\
  aa & Aa & \begin{tabular}{@{}c@{}} $\frac{1}{2}$Aa \\ $\frac{1}{2}$aa \end{tabular} & aa \\
\end{tabular}

Using the above set of possible crosses and the probabilities $u$, $v$, and
$w$, the updated frequencies $u^\prime$, $v^\prime$, and $w^\prime$ are as
follows:

\begin{align}
  u^\prime &= u^2 + \frac{1}{4}v^2 + uv \\
  v^\prime &= 2uw + uv + wv + \frac{1}{2}v^2 \\
  w^\prime &= w^2 + \frac{1}{4}v^2 + vw
\end{align}

In Hardy-Weinberg equilibrium, the genotype frequencies are equal to the
product of allele frequencies. The allele frequencies are:

\begin{align}
  p_A &= u + \frac{1}{2}v \\
  p_a &= w + \frac{1}{2}v
\end{align}

Using expressions for $p_A$ and $p_a$, the HW frequencies for $u$, $v$, and $w$ are expected to be:

\begin{align}
u_{HW} &= p_A^2 \\
  &= (u + \frac{1}{2}v)^2 \\
  &= u^2 + uv + \frac{1}{4}v^2 \\
  v_{HW} &= 2p_A p_a \\
  &= 2(u + \frac{1}{2}v)(w + \frac{1}{2}v) \\
  &= 2(uw + \frac{1}{2}vw + \frac{1}{2}vu + \frac{1}{4}v^2) \\
  &= 2uw + vw + vu + \frac{1}{2}v^2 \\
w_{HW} &= p_a^2 \\
  &= (w + \frac{1}{2}v)^2 \\
  &= w^2 + wv + \frac{1}{4}v^2
\end{align}

Note that $u^\prime = u_{HW}$, $v^\prime = v_{HW}$, and $w^\prime = w_{HW}$, so
the offspring population is in Hardy-Weinberg equilibrium.

The allele frequencies for the offspring are as follows:

\begin{align}
  p_A^\prime &= u^\prime + \frac{1}{2}v^\prime \\
  &= u^2 + uv + \frac{1}{4}v^2 + \frac{1}{2} (2uw + vw + vu + \frac{1}{2}v^2) \\
  &= u^2 + uv + \frac{1}{4}v^2 + uw + \frac{1}{2} vw + \frac{1}{2} vu + \frac{1}{4}v^2 \\
  &= u(v+w+u) + v(\frac{1}{2} v +\frac{1}{2}w + \frac{1}{2}u) \\
  &= u + \frac{1}{2} v \\
  &= p_A \\
  p_a^\prime &= w^\prime + \frac{1}{2}v^\prime \\
  &= w^2 + wv + \frac{1}{4}v^2 + \frac{1}{2} (2uw + vw + vu + \frac{1}{2}v^2 ) \\
  &= w^2 + wv + \frac{1}{4}v^2 + uw + \frac{1}{2} vw + \frac{1}{2} vu + \frac{1}{4}v^2 \\
  &= w(v+w+u) + v(\frac{1}{2} v +\frac{1}{2}w + \frac{1}{2}u) \\
  &= w + \frac{1}{2} v \\
  &= p_a
\end{align}

Therefore, the allele frequencies remain unchanged.

\textbf{4.} Suppose a population of 2000 chromosomes; 1000 carry an A allele at
a marker and 1000 carry a. Now suppose a disease mutation ($+$) arises on one
chromosome bearing an A allele, and all the rest of the chromosomes have ($-$) at
that location.

\indent\textbf{a)} What are the marginal frequencies at the marker and Disease
Susceptibility Locus (DSL)?

Markers: $p_A = .5$ and $p_a = (1-p_A) = .5$

Disease locus: $p_+ = \frac{1}{2000} = .0005$ and $p_- = (1-p_+) = \frac{1999}{2000} = .9995$

\textbf{b)} Fill in the 2x2 table of marker and disease mutation haplotypes.

\begin{tabular}{ l | c c }
      & A    & a \\ \hline
  $+$ & A$+$ & a$+$ \\
  $-$ & A$-$ & a$-$ \\
\end{tabular}

\textbf{c)} What is D for this table?

$D = p_{A+} - p_{A}p_{+} = \frac{1}{2000} - \frac{1}{2}\frac{1}{2000} = .00025$

\textbf{d)} What is the correlation between the marker and DSL?

% $r^2 = \frac{Cov(A,+)}{Var(A)Var(+)} = \frac{\frac{1}{1000}\big((1-.001)(1-.5) + (1000-1)(0-.001)(0-.5) \big)}{(.5)(1-.5)(.001)(1-.001)}$
% $r^2 = 0.000999$
% $r = 0.03160696$

$r = \frac{D}{\sqrt{p_A p_a p_+ p_-}} = \frac{.00025}{\sqrt{.5 * .5 * .0005 * .9995}} = 0.02236627$

\textbf{e)} Repeat the questions above, now assuming only 100 chromosomes, one
mutation on the same haplotype as an A allele, and a 50/50 split of A and a
alleles.

Markers: $p_A = .5$ and $p_a = (1-p_A) = .5$

Disease locus: $p_+ = .005$ and $p_- = (1-p_+) = .995$

$D = p_{A+} - p_{A}p_{+} = \frac{1}{200} - \frac{1}{2}\frac{1}{200} = .0025$

$r = \frac{D}{\sqrt{p_A p_a p_+ p_-}} = \frac{.0025}{\sqrt{.5 * .5 * .005 * .995}} = 0.07088812$

\essaysection{Lab 1}
\vspace{.5em}

\textbf{1.} What command would give you a detailed but human-readable list of the files
in a directory in reverse-order of when they have last been modified?

\begin{minted}{bash}
ls -rmathl
\end{minted}


\textbf{2.} Step 2 is composed of multiple commands. Use the descriptions of the
commands (Section 1.1) and the man command to describe Step 2
command-by-command.

\textit{cat}: print the contents of a file to the terminal \\
\textit{|}: pipe the output of the previous command to the next one \\
\textit{sort}: sort a text stream \\
\hspace{.5em} -k: sort by value (k2 $\rightarrow$ 2nd column) \\
\hspace{.5em} -n: sort numerically \\
\hspace{.5em} -r: reverse output \\
\textit{head}: show the first 10 lines of output \\
\textit{>}: redirect standard output somewhere else \\


\textbf{3.} How would you modify this command if you only wanted the top three?

Use \mintinline{bash}{head -n3} instead.

\begin{minted}{bash}
cat example.chr22.map | sort -k2nr | head -n3 > top_ten.map
\end{minted}


\textbf{4.} Find the golden nugget! Write a new script named extract.sh where the first
line of each of the top\_ten replicate files is written into a single file named
extract.txt. Note: you may want to refer to Section 1.1 to find a command that
can select single lines.

\inputminted[linenos,numbersep=.5em]{bash}{extract.sh}


\textbf{5.} What is the difference between using one versus two question marks?

\textit{?<command>}: opens the standard documentation for a object/method \\
\textit{??<string>}: do a regex match of <string> against the names and
descriptions of defined objects/methods \\


\textbf{6.} At first glance the current plot may be a bit misleading because of the
differences in scale between the axes. Use the query function learned in
Section 2.2 to fix the scale on the axes so that both axis range from 0 to 100.
Note that the necessary arguments may take vectors.

\begin{minted}{r}
plot(new_data$Bob, new_data$Steve, main="Cheeseburgers Eaten",
     xlab="Bob", ylab="Steve", xlim=c(0,100), ylim=c(0,100))
\end{minted}


\textbf{7.} You want a quick estimate on how often you will win the dice game craps on
the first roll. You do this by constructing 500 random rolls (2 dice/roll) and
looking at the results. Use the sample function to sample the numbers 1 to 6
with replacement 1000 times, naming this vector ``dice''. Take this vector and
make it into a matrix named ``pairs'' with 2 columns and 500 rows, simulating the
500 rolls. Make a new vector summing the first and second rows named ``rolls''.
If 7 and 11 are the two rolls for victory on the first roll, estimate the
probability of winning from your 500-roll sample. Hint: The function table
might be useful in this problem.

\begin{minted}[linenos,numbersep=.5em]{r}
set.seed(123)
dice = sample(1:6, 1000, replace=TRUE)
pairs = matrix(dice, nrow=500, ncol=2)
rolls = pairs[,1] + pairs[,2]
sum(rolls == 11 | rolls == 7) / length(rolls)

[1] 0.204
\end{minted}


\textbf{8.} In the for-loop that accomplishes Steps 3 \& 4, why don't we have to specify
the mean when we are filling the object ``mean\_zero''? What standard deviation
did we set for these normally-distributed values?

\mintinline{r}{rnorm} by default assumes mean of 0 and standard deviation of 1.


\textbf{9.} How would you write a command to take the standard deviation of all the
replicate means from the ``all\_means'' array?

To compute the standard deviation of means for each sampling distribution separately:
\begin{minted}[linenos,numbersep=.5em]{r}
sd_zero = sd(all_means[1,])
sd_one = sd(all_means[2,])
\end{minted}

If instead you want the standard deviation of all the means computed from both
distributions:
\begin{minted}[linenos,numbersep=.5em]{r}
sd = sd(c(all_means[1,], all_means[2,]))
\end{minted}

\end{essaystyle}
\end{document}
